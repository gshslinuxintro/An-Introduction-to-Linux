\chapter{Introduction}

\graphicspath{{./chap1/images/}}   
\textbf{서버는 모두가 함께 사용하는 것입니다.}

\section{Motivation}
경기과학고등학교(\acs{GSHS})에는 고성능 \acs{GPU} 6개, 그리고 40개의 \acs{CPU} 코어를 가진 서버가 있다. 이 서버는 강력하기에 학생들은 자신의 연구에 큰 도움을 받고 있다. 그러나, 그 서버의 무분별한 사용과 관리자 부재로 인해 연구활동에 어려움을 겪게 되었으며 신입생들의 경우 구전으로 전해지는 사용 방법에 의존해야 했다. 이에 문제의식을 느낀 37기 정보과 일동은 경기과학고등학교 리눅스 사용자 협회, 그리고 교내 봉사모둠인 GM을 만들어 서버 사용 및 관리를 체계화하는 한편 본 설명서를 통해 처음 서버를 이용하는 학생들을 돕고자 한다.

\section{Ubuntu}
Ubuntu Linux는 윈도우와 같은 운영체제로, 윈도우와 달리 무료이며 공개 소스이다. 서버에 일반적으로 이용하는 Ubuntu Server는 \acs{GUI}가 없는 Ubuntu로, 마우스 없이 명령줄만으로 제어하는 \acs{CLI} 환경이다. 윈도우의 명령 프롬프트(cmd)와 유사하다.

%\subsection{Showing the Use of Acronyms}

%In the early nineties, \acs{GSM} was deployed in many European countries. \ac{GSM} offered for the first time international roaming for mobile subscribers. The \acs{GSM}’s use of \ac{TDMA} as its communication standard was debated at length. And every now and then there are big discussion whether \ac{CDMA} should have been chosen over \ac{TDMA}.

%If you want to know more about \acf{GSM}, \acf{TDMA}, \acf{CDMA} and other acronyms, just read a book about mobile communication. Just to mention it: There is another \ac{UA}, for testing.